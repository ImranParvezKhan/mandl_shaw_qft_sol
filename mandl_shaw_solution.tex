\documentclass[letterpaper,12pt]{article}
\usepackage[letterpaper, left=0.8in, right=0.8in, top=1in, bottom=1in]{geometry}
\usepackage[titletoc]{appendix}
\usepackage[utf8]{inputenc}
\usepackage{xcolor}
\usepackage{enumerate}
\usepackage{multicol}
\usepackage{tikz}
\usepackage{amsmath,amssymb}
\usepackage{setspace}
\usepackage{natbib}
\usepackage{graphicx}
\usepackage[justification=centering]{caption}
\usepackage{subcaption}
\usepackage{hyperref}
\usepackage{mathrsfs}
\usepackage{calligra}
\usepackage{float}
\usepackage{physics}
\usepackage{slashed}
\usepackage{tikz-feynman}
\usepackage{tgtermes}
\usepackage{titlesec}
\usepackage{changepage}
\usepackage{tabto}

\setlength{\parindent}{0.4in}

\newcommand{\linia}{\rule{\linewidth}{0.5pt}}

% my own titles
\title{
	\huge{Solution to Problems in\\Quantum Field Theory}\\
	\vspace{0.1in}
	\Large{by Franz Mandl \& Graham Shaw}
}

\author{Sanha Cheong}
\def\email{\texttt{\href{mailto:sanha@stanford.edu}{sanha@stanford.edu}}}
\def\institution{Stanford University}

\date{\today}

\makeatletter
\renewcommand{\maketitle}{
\begin{center}
\textsc{\@title}
\end{center}
\vspace{2ex}
\Large{\@author} \hfill \Large{\@date}		\\
\large{\email}	\hfill	$ $			\\
\large{\institution}
\\
\linia
}
\makeatother
%%%

% custom footers and headers
\usepackage{fancyhdr,lastpage}
\pagestyle{fancy}
\lhead{}
\chead{}
\rhead{}
\cfoot{}
\rfoot{Page \thepage\ /\ \pageref*{LastPage}}
\renewcommand{\headrulewidth}{0pt}
\renewcommand{\footrulewidth}{0pt}
%

% custom section/problems
\titleformat{\section}[block]
{\vspace{0.3in}\LARGE\bfseries\centering}
{\thesection}{1em}{}

\renewcommand{\thesubsection}{\arabic{subsection}}
\titleformat{\subsection}[runin]
{\vspace{1ex}}
{\thesubsection.}{0em}{}

\renewcommand{\thesubsubsection}{(\roman{subsubsection})}
\titleformat{\subsubsection}[runin]
{\vspace{-1ex}}
{\thesubsubsection}{0em}{}

\newenvironment{problem}{\subsection{}\begin{adjustwidth}{0.25in}{}\vspace{-\baselineskip}}{\end{adjustwidth}}

\newenvironment{subproblem}{\subsubsection{}\begin{adjustwidth}{0.4in}{}\vspace{-\baselineskip}}{\end{adjustwidth}}
%

% mathematical & typeset commands
\setlength{\fboxsep}{0pt}
\setlength{\fboxrule}{.1pt}
\makeatletter
\newcommand*\dotp{\mathpalette\bigcdot@{.5}}
\newcommand*\bigcdot@[2]{\mathbin{\vcenter{\hbox{\scalebox{#2}{$\m@th#1\bullet$}}}}}
\makeatother
\newcommand{\fig}[2]{ 
	\begin{figure}[bth] 
		\centering
		\includegraphics[width = 13cm]{#1} 
		\caption {#2} 
\end{figure}}
\newcommand{\der}[2]{\frac{\diff{#1}}{\diff{#2}}}
\newcommand{\nder}[3]{\frac{\diff[#1]{#2}}{\ndiff[#1]{#3}}}
\newcommand{\pder}[2]{\frac{\partial #1}{\partial #2}}
\newcommand{\npder}[3]{\frac{\partial^#1 #2}{\partial #3^#1}}
\newcommand{\lagr}{\mathscr{L}}
\newcommand{\hamil}{\mathscr{H}}
\DeclareMathOperator{\dalem}{\Box}
\makeatletter
\def\diff{\@ifnextchar[{\@with}{\@without}}
\def\@with[#1]#2{\textrm{d}^#1#2}
\def\@without#1{\textrm{d}#1}
\makeatother
\newcommand{\define}{\equiv}
\newcommand{\done}{\tag*{$\blacksquare$}}
\newcommand{\overbar}[1]{
	\mkern 1.5mu \overline{\mkern-1.5mu\raisebox{0pt}[\dimexpr\height+0.5mm\relax]{$#1$}\mkern-1.5mu}\mkern 1.5mu
}
\newcommand{\timep}[1]{\mathrm{T}\left\{#1\right\}}
%

\bibliographystyle{abbrv}

%%%----------%%%----------%%%----------%%%----------%%%

\begin{document}

\maketitle

\section{Photons and the Electromagnetic Field}

\begin{problem}
The free-radiation field inside a cubic enclosure is given by the state:
\begin{equation*}
\ket{c} = \exp\left(-\frac{1}{2}\abs{c}^2\right)
\sum_{n=0}^{\infty} \frac{c^n}{\sqrt{n!}} \ket{n}
\end{equation*}
where $c=\abs{c}e^{i\delta}$ is any complex number and $\ket{n}$ is the state with $n$ photons with some specific wavevector $\vb{k}$ and transverse polarization $r$, omitted in the expression.

(Such a state $\ket{c}$ is called \emph{a coherent state} and is extremely useful in quantum optics and other bosonic quantum field theory.)

\begin{subproblem}
	$\ket{c}$ is normalized: $\braket{c}{c}=1$.
	\begin{align*}
		\braket{c}{c}
		&= \exp\left(-\abs{c}^2\right)
		\left(\sum_{m=0}^{\infty}\bra{m}\frac{{c^*}^m}{\sqrt{m!}}\right)
		\left(\sum_{n=0}^{\infty}\frac{c^n}{\sqrt{n!}}\ket{n}\right)	\\
		&= e^{-\abs{c}^2}
		\sum_{m=0}^{\infty}\sum_{n=0}^{\infty}
		\frac{\abs{c}^2 e^{i(n-m)\delta}}{\sqrt{m!}\sqrt{n!}}\braket{m}{n}	\\
	\intertext{At this point, we invoke the orthonormality $\braket{m}{n}=\delta_{mn}$ to obtain: }
		&= e^{-\abs{c}^2}
		\sum_{m=0}^{\infty}\sum_{n=0}^{\infty}
		\frac{\abs{c}^2 e^{i(n-m)\delta}}{\sqrt{m!}\sqrt{n!}} \delta_{mn}	\\
		&= e^{-\abs{c}^2} \sum_{n=0}^{\infty} \frac{\abs{c}^2}{n!}	\\
		&= e^{-\abs{c}^2} e^{\abs{c}^2}	\\
		&=1		 
	\end{align*}
\end{subproblem}

\begin{subproblem}
	$\ket{c}$ is an eigenstate of $a_r(\vb{k})$ with the complex eigenvalue $c$: $a_r(\vb{k}) \ket{c} = c\ket{c}$.
	\begin{align*}
		a_r(\vb{k}) \ket{c} &=
		\exp\left(-\frac{1}{2}\abs{c}^2\right)
		\sum_{n=0}^{\infty} \frac{c^n}{\sqrt{n!}} a_r(\vb{k})\ket{n}	\\
		&= \exp\left(-\frac{1}{2}\abs{c}^2\right)
		\left[0 + \sum_{n=1}^{\infty} \frac{c^n}{\sqrt{n!}} \sqrt{n}\ket{n-1}\right]	\\
	\intertext{If we redefine the index by $m=n-1$, we get:}
		&= \exp\left(-\frac{1}{2}\abs{c}^2\right)
		\sum_{m=0}^{\infty} \frac{c^{m+1}}{\sqrt{m!}} \ket{m}	\\
		&= c\ket{c}		 
	\end{align*}
\end{subproblem}

\begin{subproblem}
	The expected number of photons in the enclosure $\overbar{N}\define\ev{N}{c}$ is equal to $\abs{c}^2$.
	\begin{align*}
		\overbar{N} &\define \ev{N}{c}		\\
		&= \ev{a^\dagger a}{c}	\\
		&= \abs\Big{a\ket{c}}^2	\\
		\intertext{Since $a\ket{c}=c\ket{c}$ as shown in (ii), we get:}
		&= \abs\Big{c\ket{c}}^2	\\
		&= \abs{c}^2		 
	\end{align*}
\end{subproblem}

\begin{subproblem}
	The RMS fluctuation of the photon number is given by:
	\begin{equation*}
		\left(\Delta N\right)^2 \define \ev{N^2}{c} - \overbar{N}^2 = \abs{c}^2
	\end{equation*}
	Recall that $\left[a_r(\vb{k}),a^\dagger_s(\vb{k}')\right]=\delta_{rs}\delta_{\vb{k}\vb{k}'}$. Hence, in our case with some specific wavevector $\vb{k}$ and polarization $r$, $\left[a, a^\dagger\right]=1$. Then,
	\begin{align*}
		N^2 &= \left(a^\dagger a\right) \left(a^\dagger a\right)	\\
		&= a^\dagger (aa^\dagger) a	\\
		&= a^\dagger (N + 1) a	\\
		&= a^\dagger N a + N
	\end{align*}
	Then, the RMS fluctuation becomes:
	\begin{align*}
		\left(\Delta N\right)^2 &\define \ev{N^2}{c} - \overbar{N}^2	\\
		&= \ev{a^\dagger N a  + N}{c} - \overbar{N}^2	\\
		&= \ev{a^\dagger N a}{c} + \overbar{N} - \overbar{N}^2	\\
		&= \abs{c}^2 \ev{N}{c} + \overbar{N} - \overbar{N}^2	\\
		&= \abs{c}^4 + \abs{c}^2 - \abs{c}^4	\\
		&= \abs{c}^2		 
	\end{align*}
\end{subproblem}

\begin{subproblem}
	The expected value of the electric field $\vb{E}$ is given by:
	\begin{equation*}
		\overbar{\vb{E}} \define \ev{\vb{E}}{c} = -\vb*{\varepsilon}_r(\vb{k}) 2\sqrt{\frac{\hbar \omega_{\vb{k}}}{2V}} \abs{c} \sin(\vb{k} \dotp \vb{x} - \omega_{\vb{k}} t + \delta)
	\end{equation*}
	where $V$ is the volume of the cubic enclosure.
	
	Since we are only dealing with free radiation (i.e. no charge distribution, $\phi=0$), we can let the longitudinal field $\vb{E}_\text{L}$ be \emph{zero} in the Coulomb gauge. Then, the remaining electric field is purely transverse, which is given by:
	\begin{align*}
		\vb{E}\left(\vb{x},t\right)
		&= -\frac{1}{c} \pder{\vb{A}\left(\vb{x},t\right)}{t}	\\
		&= i \vb*{\varepsilon}_r(\vb{k}) \sqrt{\frac{\hbar \omega_{\vb{k}}}{2V}}
		\left[a_r(\vb{k})e^{i\left(\vb{k}\dotp\vb{x} - \omega_{\vb{k}}t\right)} - a^\dagger_r(\vb{k})e^{-i\left(\vb{k}\dotp\vb{x} - \omega_{\vb{k}}t\right)}\right]
	\end{align*}
	
	At this point, we define the phase of the electric field: $\theta \define \vb{k}\dotp\vb{x} - \omega_{\vb{k}}t$.
	
	The mean electric field $\overbar{\vb{E}}$ is then:
	\begin{align*}
		\overbar{E} &\define \ev{\vb{E}}{c}	\\
		&= i\vb*{\varepsilon}_r(\vb{k}) \sqrt{\frac{\hbar\omega_{\vb{k}}}{2V}}
		\left[\ev{a_r(\vb{k})}{c}e^{i\theta} - \ev{a^\dagger_r(\vb{k})}{c}e^{-i\theta}\right]	\\
	\intertext{From (ii), $a_r(\vb{k})\ket{c} = c\ket{c}$ and, similarly, $\bra{c}a^\dagger_r(\vb{k}) = \bra{c}c^*$. Also, recall that $c=\abs{c}e^{i\delta}$. Hence, we get:}
		&= i\vb*{\varepsilon}_r(\vb{k}) \sqrt{\frac{\hbar\omega_{\vb{k}}}{2V}}
		\left[ce^{i\theta} - c^*e^{-i\theta}\right]	\\
		&= i\vb*{\varepsilon}_r(\vb{k}) \sqrt{\frac{\hbar\omega_{\vb{k}}}{2V}}
		\left[2i\Im{ce^{i\theta}}\right]	\\
		&= -\vb*{\varepsilon}_r(\vb{k}) 2\sqrt{\frac{\hbar \omega_{\vb{k}}}{2V}} \abs{c} \sin(\vb{k} \dotp \vb{x} - \omega_{\vb{k}} t + \delta)		 
	\end{align*}
\end{subproblem}

\begin{subproblem}
	The RMS fluctuation $\Delta E$ in the electric field is given by:
	\begin{equation*}
		\left(\Delta E\right)^2 \define \ev{E^2}{c} - \ev{E}{c}^2 = \frac{\hbar\omega_{\vb{k}}}{2V}
	\end{equation*}
	Note that the polarization vector $\vb*{\varepsilon}_r$ has been omitted and the electric field here is treated as a scalar quantity, since we are dealing with a definite pure polarization.
	
	First, we compute the operator $E^2$:
	\begin{align*}
		E^2 &\define EE	\\
		&= \left[i\sqrt{\frac{\hbar\omega_{\vb{k}}}{2V}}
		\left(ae^{i\theta} - a^\dagger e^{-i\theta}\right)\right]^2	\\
		&= -\frac{\hbar\omega_{\vb{k}}}{2V}\left[a^2 e^{i2\theta} + a^{\dagger2}e^{-i2\theta} - aa^\dagger - a^\dagger a\right]	\\
		&= -\frac{\hbar\omega_{\vb{k}}}{2V}\left[a^2 e^{i2\theta} + a^{\dagger2}e^{-i2\theta} - 2a^\dagger a - 1\right]
	\end{align*}
	
	Then, the RMS fluctuation $\Delta E$ is:
	\begin{align*}
		\left(\Delta E\right)^2 &\define
		\ev{E^2}{c} - \ev{E}{c}^2	\\
		&= -\frac{\hbar\omega_{\vb{k}}}{2V}\left[e^{i2\theta}\ev{a^2}{c} + e^{-i2\theta}\ev{a^{\dagger2}}{c} - 2\ev{a^\dagger a}{c} - \braket{c}\right]	\\
		&\qquad - 4\frac{\hbar\omega_{\vb{k}}}{2V}\abs{c}^2\sin^2\left(\theta + \delta\right)	\\
		&= -\frac{\hbar\omega_{\vb{k}}}{2V}\left[c^2e^{i2\theta} + c^{*2}e^{-i2\theta} - 2\abs{c}^2 - 1 + 4\abs{c}^2\sin^2\left(\theta+\delta\right) \right]	\\
		&= \frac{\hbar\omega_{\vb{k}}}{2V}
		\left[1 - c^2e^{i2\theta} - \left(c^2e^{i2\theta}\right)^* + 2\abs{c}^2 - 4\abs{c}^2\sin^2\left(\theta+\delta\right)\right]	\\
		&= \frac{\hbar\omega_{\vb{k}}}{2V}
		\left[1 - \abs{c}^2\left(2\Re{e^{i2(\theta+\delta)}} - 2 + 4\sin^2(\theta+\delta)\right) \right]	\\
		&= \frac{\hbar\omega_{\vb{k}}}{2V}
		\left[1 - \abs{c}^2\left(2\cos^2(\theta+\delta)-2\sin^2(\theta+\delta) - 2 + 4\sin^2(\theta+\delta)\right)\right]	\\
		&= \frac{\hbar\omega_{\vb{k}}}{2V}		 
	\end{align*}
\end{subproblem}

From the above calculations, we observe that, in the coherent state $\ket{c}$ which is a superposition of states with all possible number of photons (we sum from $n=0$ to $n=\infty$), the relative fluctuations in both the mean number of photons and the mean electric field strength vanish:
\begin{gather*}
	\frac{\Delta N}{\overbar{N}} = \abs{c}^{-1} = \overbar{N}^{-1/2}	\\[0.5ex]
	\frac{\Delta E}{\overbar{E}} \propto \abs{c}^{-1} = \overbar{N}^{-1/2}
\end{gather*}
as $\overbar{N} \rightarrow \infty$. Hence, in this limit, the state $\ket{c}$ behaves like a classical electromagnetic field. In other words, classical beam of electromagnetic radiation is retained from quantum electrodynamics by considering a \emph{coherent state} in the limit of infinite mean number of photons.
\end{problem}



\begin{problem}
The Lagrangian of a point particle of mass $m$ and electric charge $q$ moving in an electromagnetic potential $(\phi,\vb{A})$ is given by:
\begin{equation*}
	L\left(\vb{x},\dot{\vb{x}}\right) = \frac{1}{2}m\dot{\vb{x}}^2 + \frac{q}{c}\vb{A}\dotp\dot{\vb{x}} - q\phi
\end{equation*}

\begin{subproblem}
	The momentum $\vb{p}$ conjugate to $\vb{x}$ is:
	\begin{equation*}
		\vb{p} \define \pder{L}{\dot{\vb{x}}} = m\dot{\vb{x}} + \frac{q}{c}\vb{A}
	\end{equation*}
	
	The Euler-Lagrange equations become:
	\begin{align*}
		\der{}{t}\left(\pder{L}{\dot{\vb{x}}}\right) - \pder{L}{\vb{x}} &= \der{}{t}\left(m\dot{x_i} + \frac{q}{c}A_i\right) - \frac{q}{c}\pder{(A_j\dot{x}_j)}{x_i} + q\pder{\phi}{x_i}	\\
		&= m\der{}{t}\dot{x_i} + \frac{q}{c}\left[\pder{A_i}{t} + \dot{x}_j\pder{A_i}{x_j}\right] - \frac{q}{c}\pder{A_j}{x_i}\dot{x_j} + q\pder{\phi}{x_i}	\\
		&= m\der{}{t}\dot{x_i} + q\left(-\frac{1}{c}\pder{A_i}{t} - \pder{\phi}{x_i}\right) + \frac{q}{c}\dot{x}_j\left(\pder{A_i}{x_j} - \pder{A_j}{x_i}\right)	\\
		&= m\der{}{t}\dot{\vb{x}} + q\left(-\frac{1}{c}\pder{\vb{A}}{t}-\pder{\phi}{\vb{x}}\right) + \frac{q}{c}\dot{\vb{x}}\dotp\left(\curl\vb{A}\right)
	\end{align*}
	where we have used the index notation; repeated indices imply summation.
	
	The electromagnetic fields $\vb{E}, \vb{B}$ are described in terms of the potentials as:
	\begin{equation*}
		\vb{E} = -\pder{\phi}{\vb{x}} - \frac{1}{c}\pder{\vb{A}}{t} \,, \,
		\vb{B} = \curl \vb{A}
	\end{equation*}
	and therefore the Euler-Lagrange equations give:
	\begin{equation*}
		m\der{}{t}\dot{\vb{x}} = q\left(\vb{E} + \frac{1}{c}\dot{\vb{x}}\dotp\vb{B}\right)		 
	\end{equation*}
\end{subproblem}

\begin{subproblem}
	Given a Lagrangian $L\left(\vb{x},\dot{\vb{x}}\right)$, the Hamiltonian $H(\vb{x},\vb{p})$ is:
	\begin{equation*}
		H(\vb{x},\vb{p}) = \dot{\vb{x}}\dotp\vb{p} - L
	\end{equation*}
	
	Hence, in our case, the Hamiltonian becomes:
	\begin{align*}
		H(\vb{x},\vb{p}) &= \dot{\vb{x}}\dotp\left(m\dot{\vb{x}}+\frac{q}{c}\vb{A}\right) - \left(\frac{1}{2}m\dot{\vb{x}}^2 + \frac{q}{c}\vb{A}\dotp\dot{\vb{x}} - q\phi\right)	\\
		&= \frac{1}{2}m\dot{\vb{x}}^2 + q\phi	\\
	\intertext{and since $m\dot{\vb{x}} = \vb{p} - \frac{q}{c}\vb{A}$, we get the desired Hamiltonian:}
		&= \frac{1}{2m}\left(\vb{p} - \frac{q}{c}\vb{A}\right)^2 + q\phi		 
	\end{align*}
	
	With this Hamiltonian, the Hamiltonian equations of motion give:
	\begin{align*}
		\dot{\vb{x}} &= \pder{H}{\vb{p}}	\\
		&= \frac{1}{m}\left(\vb{p} - \frac{q}{c}\vb{A}\right)
	\end{align*}
	from which we retain:
	\begin{equation*}
		\vb{p} = m\dot{\vb{x}} + \frac{q}{c}\vb{A}
	\end{equation*}
	and also
	\begin{align*}
		\dot{\vb{p}} &= m\der{}{t}\dot{x_i} + \frac{q}{c}\left(\pder{A_i}{t} + \dot{x_j}\pder{A_i}{x_j}\right)	\\
		&= -\pder{H}{\vb{x}}	\\
		&= -\frac{1}{m}\left(p_j-\frac{q}{c}A_j\right)
		\pder{}{x_i}\left(-\frac{q}{c}A_j\right) 
		- q\pder{\phi}{x_i}		\\
		&= +\frac{q}{c}\dot{x}_j\pder{A_j}{x_i} - q\pder{\phi}{x_i}
	\end{align*}
	from which we retain:
	\begin{equation*}
		m\der{}{t}\dot{\vb{x}} = q\left(\vb{E}+\frac{1}{c}\dot{\vb{x}}\dotp\vb{B}\right)	 
	\end{equation*}
\end{subproblem}
\end{problem}



\begin{problem}
Consider \textit{a Thomson Scattering} process such that an unpolarized photon with a wavevector $\vb{k}$ \textit{collides} with an electron, and a new photon with a wavevector $\vb{k}'$ linearly polarized in a specific direction is emitted at an angle $\theta$ relative to the incoming photon ($\hat{\vb{k}}\dotp\hat{\vb{k}}'=\cos\theta$).

The differential cross-section for a Thomson scattering of photons with definite initial and final polarizations $\alpha, \beta$ is given by:
\begin{equation*}
	\sigma_{\alpha\rightarrow\beta} \, \diff{\vb*{\Omega}}
	= r_0^2 \left[\vb*{\varepsilon}_\alpha\dotp\vb*{\varepsilon}'_\beta\right]^2 \, \diff{\vb*{\Omega}}
\end{equation*}
where $\vb*{\varepsilon}_\alpha, \vb*{\varepsilon}'_\beta$ are the initial and final polarization vectors respectively, and $r_0 \define \frac{e^2}{4\pi mc^2}$ is the classical electron radius.

To apply this result to our case, we average over the initial polarization $\alpha$ and fix a definite final polarization, say $\beta=1$. That is,
\begin{align*}
	\sigma_{\text{unp.}\rightarrow 1} \, \diff{\vb*{\Omega}}
	&= \frac{1}{2}\sum_{\alpha=1,2} \sigma_{\alpha\rightarrow 1} \, \diff{\vb*{\Omega}}
\end{align*}

Since $\vb*{\varepsilon}_1, \vb*{\varepsilon}_2$, and $\hat{\vb{k}}$ form an orthonormal coordinate system,
\begin{equation*}
	\sum_{\alpha=1,2} \left(\vb*{\varepsilon}_\alpha \dotp \vb*{\varepsilon}'_\beta\right)^2
	= 1 - \left(\hat{\vb{k}} \dotp \vb*{\varepsilon}'_\beta\right)^2
\end{equation*}
Similarly, $\vb*{\varepsilon}'_1, \vb*{\varepsilon}'_2$, and $\hat{\vb{k}}'$ also form an orthonormal coordinate system. In this coordinate system, we can write:
\begin{equation*}
	\hat{\vb{k}} = \left(\sin\theta\cos\phi, \sin\theta\sin\phi, \cos\theta\right)
\end{equation*}

Therefore, we obtain:
\begin{align*}
	\sigma_{\text{unp.}\rightarrow 1} \, \diff{\vb*{\Omega}}
	&= \frac{1}{2}\sum_{\alpha=1,2} r_0^2 \left[\vb*{\varepsilon}_\alpha \dotp \vb*{\varepsilon}'_1\right]^2  \, \diff{\vb*{\Omega}}	\\
	&= \frac{1}{2}r_0^2\left(1 - \sin^2\theta\cos^2\phi\right) \, \diff{\vb*{\Omega}}
\end{align*}

If we consider both of the two perpendicular final polarizations, we obtain the unpolarized differential cross-section:
\begin{align*}
	\sigma_{\text{unp.}} \, \diff{\vb*{\Omega}} &=
	\left(\sigma_{\text{unp.}\rightarrow 1} + \sigma_{\text{unp.}\rightarrow 2}\right) \, \diff{\vb*{\Omega}}	\\
	&= \frac{1}{2}r_0^2 \left(2-\sin^2\theta\cos^2\phi-\sin^2\theta\sin^2\phi\right) \, \diff{\vb*{\Omega}}	\\
	&= \frac{1}{2}r_0^2\left(1+\cos^2\theta\right) \, \diff{\vb*{\Omega}}		 
\end{align*}
which is identical Eq.~(1.69a).

Now, consider the case when the scattering angle $\theta = 90^\circ$. Since the choice of the final state polarization vectors $\vb*{\varepsilon}'_1, \vb*{\varepsilon}'_2$ are arbitrary within a plane perpendicular to $\hat{\vb*{k}}'$, we choose them such that:
\begin{equation*}
	\vb*{\varepsilon}'_1 = \frac{\hat{\vb{k}}\cross\hat{\vb{k}}'}{\abs{\hat{\vb{k}}\cross\hat{\vb{k}}'}}
\end{equation*}
and naturally $\vb*{\varepsilon}'_2 = \hat{\vb{k}}' \cross \vb*{\varepsilon}'_1$ for an orthonormal basis. By construction, $\hat{\vb{k}} = \left(\cos\phi, \sin\phi, 0\right)$ is perpendicular to $\vb*{\varepsilon}'_1$ and therefore $\phi=\pi/2$. Then,
\begin{align*}
	\sigma_{\alpha\rightarrow 2} = 0		 
\end{align*}
which implies that the final state polarization must be 100\% in the $\vb*{\varepsilon}'_1$ direction, which is the direction normal to the plane of scattering.

\end{problem}



\section{Lagrangian Field Theory}

\begin{problem}
Show the the transformation
\begin{equation*}
	\lagr'\left(\phi_r, \phi_{r,\alpha}\right) = \lagr\left(\phi_r,\phi_{r,\alpha}\right) + \partial_\alpha \Lambda^\alpha(x)
\end{equation*}
where $\Lambda^\alpha(x), \alpha=0, ... \,, 3$ are arbitrary functions of the fields $\phi_r(x)$ does not alter the equations of motions.

First, since $\Lambda^\alpha$ are functions of $\phi_r(x)$ only, we have:
\begin{equation*}
	\pder{\Lambda^\alpha}{x^\alpha}
	= \pder{\phi_r}{x^\alpha}
	\pder{\Lambda^\alpha}{\phi_r}
	= \phi_{r,\alpha} \pder{\Lambda^\alpha}{\phi_r}
\end{equation*}

Then, the Euler-Lagrange equations with the new Lagrangian density $\lagr'$ become:
\begin{align*}
	\pder{}{x^\alpha}\left(\pder{\lagr'}{\phi_{r,\alpha}}\right) - \pder{\lagr'}{\phi_r}
	&= \pder{}{x^\alpha}\left(\pder{\lagr}{\phi_{r,\alpha}} + \pder{}{\phi_{r,\alpha}}\left(\phi_{s,\alpha}\pder{\Lambda^\alpha}{\phi_s}\right)\right)	\\
	&\qquad - \pder{}{\phi_r}\left(\lagr + \phi_{s,\alpha}\pder{\Lambda^\alpha}{\phi_s}\right)	\\
	&= \pder{}{x^\alpha}\left(\pder{\lagr}{\phi_{r,\alpha}} + \delta_{rs}\pder{\Lambda^\alpha}{\phi_s}\right)	 - \pder{\lagr}{\phi_r} - \phi_{s,\alpha}\frac{\partial^2 \Lambda^\alpha}{\partial\phi_r \partial\phi_s}	\\
	&= \pder{}{x^\alpha}\left(\pder{\lagr}{\phi_{r,\alpha}}\right) -\pder{\lagr}{\phi_r} 	\\
	&\qquad + \pder{}{x^\alpha}\left(\pder{\Lambda^\alpha}{\phi_r}\right) - \phi_{s,\alpha}\frac{\partial^2 \Lambda^\alpha}{\partial\phi_r \partial\phi_s}	\\
	&= \pder{}{x^\alpha}\left(\pder{\lagr}{\phi_{r,\alpha}}\right) -\pder{\lagr}{\phi_r}	\\
	&\qquad + \pder{\phi_s}{x^\alpha}\frac{\partial^2 \Lambda^\alpha}{\partial\phi_s \partial\phi_r} - \phi_{s,\alpha}\frac{\partial^2 \Lambda^\alpha}{\partial\phi_r \partial\phi_s}	\\
	&= \pder{}{x^\alpha}\left(\pder{\lagr}{\phi_{r,\alpha}}\right) -\pder{\lagr}{\phi_r}	\\
	&= 0		 
\end{align*}
\end{problem}



\begin{problem}
The real Klein-Gordon field is described by the Hamiltonian density:
\begin{equation*}
	\hamil(x) = \frac{1}{2}\left(c^2\pi^2(x) + \left(\grad{\phi}\right)^2 + \mu^2\phi^2\right)
\end{equation*}
and the Hamiltonian of the field is:
\begin{equation*}
H \define \int \diff[3]{\vb{x}} \, \hamil(x)
\end{equation*}

The commutation relations of the fields are:
\begin{gather*}
	\left[\phi(\vb{x},t), \pi(\vb{x}',t)\right] = i\hbar\delta\left(\vb{x}-\vb{x}'\right)	\,,\\
	\left[\phi(\vb{x},t), \phi(\vb{x}',t)\right]
	= \left[\pi(\vb{x},t), \pi(\vb{x}',t)\right] = 0
\end{gather*}

Before computing the commutation relations involving the Hamiltonian $H$, let us first compute some simpler commutators.
\begin{align*}
	\left[\pi^2(\vb{x}',t), \phi(\vb{x},t)\right]
	&= \pi(\vb{x}',t)\pi(\vb{x}',t)\phi(\vb{x},t) 
	- \phi(\vb{x},t)\pi(\vb{x}',t)\pi(\vb{x}',t)	\\
	&= \pi(\vb{x}',t)\pi(\vb{x}',t)\phi(\vb{x},t)	\\
	&\qquad - \Big(\pi(\vb{x}',t)\phi(\vb{x},t) + \left[\phi(\vb{x},t), \pi(\vb{x}',t)\right]\Big)\pi(\vb{x}',t)	\\
	&= \pi(\vb{x}',t)\left[\pi(\vb{x}',t), \phi(\vb{x},t)\right]
	- \left[\phi(\vb{x},t), \pi(\vb{x}',t)\right]\pi(\vb{x}',t)	\\
	&= -2i\hbar \delta(\vb{x}-\vb{x}')\pi(\vb{x}',t)
\end{align*}
and
\begin{align*}
	\left[\phi^2(\vb{x'},t), \pi(\vb{x},t)\right]
	&= \phi(\vb{x}',t)\phi(\vb{x}',t)\pi(\vb{x},t)
	- \pi(\vb{x},t)\phi(\vb{x}',t)\phi(\vb{x}',t)	\\
	&= \phi(\vb{x}',t)\phi(\vb{x}',t)\pi(\vb{x},t)	\\
	&\qquad - \Big(\phi(\vb{x}',t)\pi(\vb{x},t) + \left[\pi(\vb{x},t), \phi(\vb{x}',t)\right]\Big)\phi(\vb{x}',t)	\\
	&= \phi(\vb{x}',t)\left[\phi(\vb{x}',t), \pi(\vb{x},t)\right] - \left[\pi(\vb{x},t), \phi(\vb{x}',t)\right]\phi(\vb{x}',t)	\\
	&= 2i\hbar \delta(\vb{x}'-\vb{x})\phi(\vb{x}',t)
\end{align*}

Also, define $\grad' \define \left(\pder{}{x'}, \pder{}{y'}, \pder{}{z'}\right)$. (This is to emphasize that the partial derivatives are taken with respect to $\vb{x}'$, and $\grad'$ therefore does not act on functions of $\vb{x}$, e.g. $\grad{'\phi(\vb{x},t)}=0$.) Then, we have:
\begin{equation*}
	\left[\left(\grad'{\phi(\vb{x}',t)}\right)^2, \phi(\vb{x},t)\right] = 0
\end{equation*}
and
\begin{align*}
	\left[\left(\grad'{\phi(\vb{x}',t)}\right)^2, \pi(\vb{x},t)\right] &=
	\Big(\grad'{\phi(\vb{x}',t)}\dotp\grad'{\phi(\vb{x}',t)}\Big)\pi(\vb{x},t)	\\
	&\qquad - \pi(\vb{x},t)\Big(\grad'{\phi(\vb{x}',t)}\dotp\grad'{\phi(\vb{x}',t)}\Big)	\\
	&= \grad'{\phi(\vb{x}',t)}\dotp\Big(\grad'{\phi(\vb{x}',t)\pi(\vb{x},t)}\Big)	\\
	&\qquad - \grad'{\Big(\pi(\vb{x},t)\phi(\vb{x}',t)\Big)} \dotp \grad'{\phi(\vb{x}',t)}	\\
	&= \grad'{\phi(\vb{x}',t)} \dotp \Big(\pi(\vb{x},t)\grad'{\phi(\vb{x}',t)} + \grad'{[\phi(\vb{x}',t), \pi(\vb{x},t)]}\Big)	\\
	&\qquad - \Big(\grad'{\phi(\vb{x}',t)}\pi(\vb{x},t) - \grad'{[\phi(\vb{x}',t),\pi(\vb{x},t)]}\Big) \dotp \grad'{\phi(\vb{x}',t)}	\\
	&= \grad'{\phi(\vb{x}',t)} \dotp i\hbar\grad'{\delta(\vb{x'}-\vb{x})} + i\hbar\grad{\delta(\vb{x}-\vb{x}')} \dotp \grad'{\phi(\vb{x}',t)}	\\
	&= 2i\hbar \grad'{\delta(\vb{x}-\vb{x}')} \dotp \grad'{\phi(\vb{x}',t)}
\end{align*}

Now, we simply combine the above results to compute the commutator relations with the Hamiltonian:
\begin{align*}
	\left[H, \phi(x)\right] &= \left(\int \diff[3]{\vb{x}'}\,\hamil(\vb{x}',t)\right)\phi(\vb{x},t) - \phi(\vb{x},t)\left(\int \diff[3]{\vb{x}'}\,\hamil(\vb{x}',t)\right)	\\
	&= \int \diff[3]{\vb{x}'} \,
	\frac{1}{2}\Big(c^2\pi^2(\vb{x}',t) + (\grad'{\phi}(\vb{x}',t))^2 + \mu^2\phi^2(\vb{x}',t)\Big)\phi(\vb{x},t)	\\
	&\qquad - \int \diff[3]{\vb{x}'} \,
	\phi(\vb{x},t)\frac{1}{2}\Big(c^2\pi^2(\vb{x}',t) + (\grad'{\phi}(\vb{x}',t))^2 + \mu^2\phi^2(\vb{x}',t)\Big)	\\
	&= \int \diff[3]{\vb{x}'}\,
	\frac{1}{2}c^2\left[\pi^2(\vb{x}',t), \phi(\vb{x},t)\right]	\\
	&= \int \diff[3]{\vb{x}'}\,
	\frac{1}{2}c^2 \left(-2i\hbar\delta(\vb{x}-\vb{x}')\pi(\vb{x}',t)\right)
	\\
	&= -i\hbar c^2 \pi(x)		
\end{align*}
and
\begin{align*}
	[H, \pi(x)]
	&= \int \diff[3]{\vb{x}'} \,
	\frac{1}{2}\Big(\left[\left(\grad'{\phi(\vb{x}',t)}\right)^2,\pi(\vb{x},t)\right] + \mu^2\left[\phi^2(\vb{x}',t), \pi(\vb{x},t)\right]\Big)	\\
	&= \int \diff[3]{\vb{x}'} \,
	\frac{1}{2}\Big(2i\hbar\grad'{\delta(\vb{x}-\vb{x}')}\dotp\grad'{\phi(\vb{x}',t)} + 2i\hbar\mu^2\delta(\vb{x}'-\vb{x})\phi(\vb{x}',t)\Big)	\\
	&= i\hbar(\mu^2 - \laplacian)\phi(x)		 
\end{align*}
\end{problem}

\pagebreak

\begin{problem}
Consider the Lagrangian density given by:
\begin{equation*}
	\lagr = -\frac{1}{2}\big[\partial_\alpha\phi_\beta(x)\big]\big[\partial^\alpha\phi^\beta(x)\big] +
	\frac{1}{2}\big[\partial_\alpha\phi^\alpha(x)\big]\big[\partial_\beta\phi^\beta(x)\big] +
	\frac{\mu^2}{2}\phi_\alpha(x)\phi^\alpha(x)
\end{equation*}
where $\phi^\alpha(x)$ is a real vector field.

The corresponding Euler-Lagrange equations are:
\begin{align*}
	\pder{}{x^\beta}\left(\pder{\lagr}{\phi^\alpha_{\;,\beta}}\right) - \pder{\lagr}{\phi^\alpha}
	&= \pder{}{x^\beta}\Big(-\phi_\alpha^{\;,\beta} + \delta^\beta_\alpha \partial_\gamma\phi^\gamma\Big) - \mu^2\phi_\alpha	\\
	&= -\partial_\beta \partial^\beta \phi_\alpha + \partial_\alpha \partial_\gamma \phi^\gamma - \mu^2\phi_\alpha	\\
	&= -\dalem\phi_\alpha + \partial_\alpha\partial_\beta \phi^\beta - \mu^2\phi_\alpha	\\
	&= 0
\end{align*}
Rewriting $\phi_\alpha = g_{\alpha\beta} \phi^\beta$ and rearranging, we get:
\begin{equation*}
	\left[g_{\alpha\beta}\left(\dalem +\mu^2\right) - \partial_\alpha\partial_\beta\right]\phi^\beta(x) = 0		 
\end{equation*}

Now, if we take apply an additional derivative $\partial^\alpha$ to the field equations, we get:
\begin{align*}
	\partial^\alpha\left[g_{\alpha\beta}\left(\dalem+\mu^2\right) - \partial_\alpha\partial_\beta\right]\phi^\beta(x)
	&= \partial_\beta \dalem \phi^\beta + \mu^2 \partial_\beta\phi^\beta - \dalem \partial_\beta \phi^\beta	\\
	&= \mu^2 \partial_\beta \phi^\beta	\\
	&= 0
\end{align*}
For non-zero $\mu$ (i.e. the fields are \emph{massive}), this implies \emph{the Lorenz Gauge} condition:
\begin{equation*}
	\partial_\alpha \phi^\alpha (x) = 0			 
\end{equation*}
\end{problem}



\begin{problem}
The 3-momentum operator of the fields is defined as:
\begin{equation*}
	P^j \define \int \diff[3]{\vb{x}} \, \pi_r(x) \pder{\phi_r(x)}{x_j}
\end{equation*}

This operator satisfies the following commutation relations:
\begin{align*}
	\left[P^j, \phi_r(x)\right] &\define P^j\phi_r(x) - \phi_r(x)P^j	\\
	&= \int \diff[3]{\vb{x}'} \, \pi_r(\vb{x}',t)\pder{\phi_r(\vb{x}',t)}{x'_j}\phi_r(\vb{x},t) - \phi_r(\vb{x,t})\pi_r(\vb{x}',t)\pder{\phi_r(\vb{x}',t)}{x'_j}		\\
\intertext{because the derivatives are with respect to $\vb{x}'$ and does not affect $\phi_r(\vb{x},t)$,}
	&= \int \diff[3]{\vb{x}'} \, \left[\pi_r(\vb{x}',t), \phi_r(\vb{x},t)\right]\pder{\phi_r(\vb{x}',t)}{x'_j}	\\
	&= -\int \diff[3]{\vb{x}'} \, i\hbar\delta\left(\vb{x}-\vb{x}'\right)\pder{\phi_r(\vb{x}',t)}{x'_j}	\\
	&= -i\hbar \pder{\phi_r(x)}{x_j}
\end{align*}
and, similarly,
\begin{align*}
	\left[P^j, \pi_r(x)\right]
	&= \int \diff[3]{\vb{x}'} \, \pi_r(\vb{x}',t)\pder{\phi_r(\vb{x}',t)}{x'_j}\pi_r(\vb{x},t) - \pi_r(\vb{x},t)\pi_r(\vb{x}',t)\pder{\phi_r(\vb{x}',t)}{x'_j}	\\
	&= \int \diff[3]{\vb{x}'} \,
	\pi_r(\vb{x}',t)\left[\pder{\phi_r(\vb{x}',t)}{x'_j}, \pi_r(\vb{x},t)\right]		\\
	&= \int \diff[3]{\vb{x}'} \,
	\pi_r(\vb{x}',t) \pder{}{x'_j}\left[\phi_r(\vb{x}',t), \pi_r(\vb{x},t)\right]		\\
	&= i\hbar \int \diff[3]{\vb{x}'} \,
	\pi_r(\vb{x}',t) \pder{}{x'_j}\delta\left(\vb{x}'-\vb{x}\right)	\\
	&= -i\hbar \pder{\pi_r(x)}{x_j}			 
\end{align*}

Consider an arbitrary operator $F(x)=F(\phi_r(x),\pi_r(x))$ that can be written as a power series of $\phi_r(x)$ and $\pi_r(x)$:
\begin{equation*}
	F(x) = \sum_{n=0}^{\infty} a_n \phi^n_r(x) + b_n \pi^n_r(x)
\end{equation*}
where $a_n$'s and $b_n$'s are arbitrary complex numbers.

Using the commutator identity
\begin{equation*}
	\left[A, B^n\right] = nB^{n-1}\left[A,B\right]
\end{equation*}
we se that $F(x)$ satisfies:
\begin{align*}
	\left[P^j, F(x)\right] &= \sum_{m,n=0}^{\infty} a_m\left[P^j, \phi^m_r(x)\right] + b_n\left[P^j, \pi^n_r(x)\right]	\\
	&= \sum_{m,n=0}^{\infty} a_m m\phi^{m-1}_r(x) \left[P^j, \phi_r(x)\right] + b_n n\pi^{n-1}_r(x) \left[P^j, \pi_r(x)\right]		\\
	&= -i\hbar\sum_{m,n=0}^{\infty} a_m m\phi^{m-1}_r(x)\pder{\phi_r(x)}{x_j} + b_n n\pi^{n-1}_r(x)\pder{\pi_r(x)}{x_j}	\\
	&= -i\hbar \pder{F(x)}{x_j}		 
\end{align*}

Define the 4-momentum of the fields as: $P^\alpha = (H/c, P^j), \alpha = 0, ... \,, 3$ where $H$ is the Hamiltonian of the fields. Then, combining the above result with the Heisenberg equation of motion:
\begin{equation*}
	[H, F(x)] = -i\hbar c \pder{F(x)}{x_0}
\end{equation*}
gives the covariant equations of motion:
\begin{equation*}
	\left[P^\alpha, F(x)\right] = -i\hbar \pder{F(x)}{x_\alpha}			 
\end{equation*}
\end{problem}



\begin{problem}
Consider a scalar field $\phi_r(x_\alpha)$ invariant under translations $x_\alpha \longrightarrow x'_\alpha = x_\alpha + \delta_\alpha$ where $\delta_\alpha$ is a constant 4-vector. i.e.,
\begin{align*}
	\phi'(x'_\alpha) &= \phi(x_\alpha)		\\
	\intertext{or, equivalently,} 
	\phi'(x_\alpha) &= \phi(x_\alpha - \delta_\alpha)
\end{align*}

In the limit of an infinitesimal transformation ($\delta_\alpha \ll 1$), we can write the transformation as:
\begin{align*}
	\phi'(x_\alpha) &= \phi(x_\alpha - \delta_\alpha)	\\
	&= \phi(x_\alpha) - \delta_\alpha \pder{\phi}{x_\alpha} + \mathcal{O}\left(\delta^2\right)	\\
	&\approx \phi(x_\alpha) + \delta_\alpha\frac{1}{i\hbar}\left[P^\alpha, \phi(x)\right]
\end{align*}
where we have used the result from the previous problem: $\left[P^\alpha, F(x)\right] = -i\hbar\pder{F(x)}{x_\alpha}$.

Let us write the corresponding unitary transformation $U$ as:
\begin{equation*}
U = e^{i\delta_\alpha T^\alpha}
\end{equation*}
where $T^\alpha$ is a Hermitian operator ($T^{\alpha\dagger}=T^\alpha$). In terms of $U$, the same translation transformation can also be written as:
\begin{align*}
	\phi'(x) &= U\phi(x)U^\dagger	\\
	&\approx (1+i\delta_\alpha T^\alpha) \phi(x) (1-i\delta_\alpha T^\alpha)	\\
	&= \phi(x) + i\delta_\alpha\left[T^\alpha, \phi(x)\right] + \mathcal{O}(\delta^2)
\end{align*}

Comparing the two results, we see that $T^\alpha = -P^\alpha / \hbar$ and therefore:
\begin{equation*}
	U = e^{-i\delta_\alpha P^\alpha / \hbar}		 
\end{equation*}

\end{problem}



\section{The Klein-Gordon Field}

\begin{problem}
A real Klein-Gordon field can be written as:
\begin{align*}
	\phi(x) &= \phi^+(x) + \phi^-(x)	\\
	&= \sum_{\vb{k}} \sqrt{\frac{\hbar c^2}{2V\omega_{\vb{k}}}} \left[a(\vb{k})e^{-ikx} + a^\dagger(\vb{k})e^{ikx}\right]
\end{align*}
where $\vb{k}$ is the wavenumber of the field and $k^0 = \omega_{\vb{k}}/c$.

Then, we have:
\begin{align*}
	\dot{\phi}(x) &\define \pder{\phi(x)}{t}	\\
	&= \sum_{\vb{k}} \sqrt{\frac{\hbar c^2}{2V\omega_{\vb{k}}}} 
	\left[\left(-i\frac{\omega_{\vb{k}}}{c}c\right)a(\vb{k})e^{-ikx} + \left(i\frac{\omega_{\vb{k}}}{c}c\right) a^\dagger(\vb{k})e^{ikx}\right]	\\
	&= -i\sum_{\vb{k}} \sqrt{\frac{\hbar c^2}{2V \omega_{\vb{k}}}} \omega_{\vb{k}}
	\left[a(\vb{k})e^{-ikx} - a^\dagger(\vb{k})e^{-ikx}\right]
\end{align*}
and therefore:
\begin{align*}
	i\dot{\phi}(x) + \omega_{\vb{k}}\phi(x)
	&= \sum_{\vb{k}} \sqrt{\frac{\hbar c^2}{2V\omega_{\vb{k}}}} \omega_{\vb{k}} 2a(\vb{k})e^{-ikx}
\end{align*}

To re-write the expression in the $\vb{k}$-space, use Fourier transform, i.e. apply $\int_{V} \diff[3]{\vb{x}} \, e^{ik'x}$ on both sides. Then, we get:
\begin{align*}
	\int_{V} \diff[3]{\vb{x}} \, e^{ik'x} \left(i\dot{\phi}(x) + \omega_{\vb{k}}\phi(x)\right)
	&= \int_{V} \diff[3]{\vb{x}} \, \sum_{\vb{k}} \sqrt{\frac{2\hbar c^2 \omega_{\vb{k}}}{V}} a(\vb{k})e^{-ikx} e^{ik'x}	\\
	&= \int_{V} \diff[3]{\vb{x}} \, \sum_{\vb{k}} \sqrt{\frac{2\hbar c^2 \omega_{\vb{k}}}{V}} a(\vb{k}) \delta_{\vb{k}\vb{k}'}	\\
	&= \sqrt{2\hbar c^2 V \omega_{\vb{k}'}} a(\vb{k}')
\end{align*}
Rewriting this, we get:
\begin{equation*}
	a(\vb{k}) = \left(2\hbar c^2 V \omega_{\vb{k}}\right)^{-1/2} \int_{V} \diff[3]{\vb{x}} \, e^{ikx}\left(i\dot{\phi}(x) + \omega_{\vb{k}}\phi(x)\right)		 
\end{equation*}



We define $x=(\vb{x},t)$ and $x'=(\vb{x}',t)$ to denote two different points in space. Then, we have the following commutator relationship:
\begin{align*}
	\left[a(\vb{k}), a^\dagger(\vb{k}')\right]
	&= \frac{1}{2\hbar c^2 V \sqrt{\omega_{\vb{k}} \omega_{\vb{k}'}}} \int_V \diff[3]{\vb{x}} \int_V \diff[3]{\vb{x}'} \, e^{i\left(kx-k'x'\right)} \\
	&\qquad \times \Big[ \left(i\dot{\phi}(x) + \omega_{\vb{k}}\phi(x)\right)\left(-i\dot{\phi}(x') + \omega_{\vb{k}'}\phi(x')\right)	\\
	&\qquad \hspace{5em} - \left(-i\dot{\phi}(x') + \omega_{\vb{k}'}\phi(x')\right)\left(i\dot{\phi}(x) + \omega_{\vb{k}}\phi(x)\right)\Big]
\end{align*}
since $\phi(x)$ is a real field and thus a Hermitian operator after quantization.

Given the commutator relationship $[\phi(\vb{x},t), \dot{\phi}(\vb{x}',t)]$, this then becomes:
\begin{align*}
	\left[a(\vb{k}, a^\dagger(\vb{k}'))\right] &= \frac{1}{2\hbar c^2 V \sqrt{\omega_{\vb{k}} \omega_{\vb{k}'}}} \int_V \diff[3]{\vb{x}} \int_V \diff[3]{\vb{x}'} \, e^{i\left(kx-k'x'\right)} \\
	&\qquad \times \Big[i\omega_{\vb{k}'}[\dot{\phi}(x), \phi(x')] - i\omega_{\vb{k}}[\phi(x), \dot{\phi}(x')]\Big]	\\
	&= \frac{1}{2\hbar c^2 V \sqrt{\omega_{\vb{k}} \omega_{\vb{k}'}}} \int_V \diff[3]{\vb{x}} \int_V \diff[3]{\vb{x}'} \, e^{i\left(kx-k'x'\right)} \\
	&\qquad \times \Big[\omega_{\vb{k}'}\hbar c^2 \delta\left(\vb{x}'-\vb{x}\right) + \omega_{\vb{k}}\hbar c^2 \delta\left(\vb{x} - \vb{x}'\right)\Big]	\\
	&= \frac{ \hbar c^2 \left(\omega_{\vb{k}}+\omega_{\vb{k}'}\right)}{2\hbar c^2 V \sqrt{\omega_{\vb{k}} \omega_{\vb{k}'}}} \int_V \diff[3]\vb{x} \,
	e^{i\left(k-k'\right)x}	\\
	&= \frac{\omega_{\vb{k}} + \omega_{\vb{k}'}}{2V\sqrt{\omega_{\vb{k}}\omega_{\vb{k}'}}} V \delta_{\vb{k}\vb{k}'} e^{i\left(\omega_{\vb{k}} - \omega_{\vb{k}'}\right)t}	\\
	&= \delta_{\vb{k}\vb{k}'}		
\end{align*}

Also, we see that:
\begin{align*}
	\left[a(\vb{k}), a(\vb{k}')\right]
	&= \frac{1}{2\hbar c^2 V \sqrt{\omega_{\vb{k}} \omega_{\vb{k}'}}} \int_V \diff[3]{\vb{x}} \int_V \diff[3]{\vb{x}'} \, e^{i\left(kx+k'x'\right)} \\
	&\qquad \times \Big[ \left(i\dot{\phi}(x) + \omega_{\vb{k}}\phi(x)\right)\left(i\dot{\phi}(x') + \omega_{\vb{k}'}\phi(x')\right)	\\
	&\qquad \hspace{5em} - \left(i\dot{\phi}(x') + \omega_{\vb{k}'}\phi(x')\right)\left(i\dot{\phi}(x) + \omega_{\vb{k}}\phi(x)\right)\Big]	\\
	&= \frac{1}{2\hbar c^2 V \sqrt{\omega_{\vb{k}} \omega_{\vb{k}'}}} \int_V \diff[3]{\vb{x}} \int_V \diff[3]{\vb{x}'} \, e^{i\left(kx+k'x'\right)} \\
	&\qquad \times \Big[i\omega_{\vb{k}'}[\dot{\phi}(x),\phi(x')] + i\omega_{\vb{k}}[\phi(x), \dot{\phi}(x')]\Big]	\\
	&= \frac{1}{2\hbar c^2 V \sqrt{\omega_{\vb{k}} \omega_{\vb{k}'}}} \int_V \diff[3]{\vb{x}} \int_V \diff[3]{\vb{x}'} \, e^{i\left(kx+k'x'\right)} \\
	&\qquad \times \Big[+\omega_{\vb{k}'}\hbar c^2\delta\left(\vb{x}'-\vb{x}\right) - \omega_{\vb{k}}\hbar c^2 \delta(\vb{x}-\vb{x}')\Big]	\\
	&= \frac{1}{2 V \sqrt{\omega_{\vb{k}} \omega_{\vb{k}'}}} \int_V \diff[3]{\vb{x}} \, e^{i\left(k+k'\right)x} \Big[\omega_{\vb{k}'} - \omega_{\vb{k}}\Big]	\\
	&= \frac{1}{2\sqrt{\omega_{\vb{k}}\omega_{\vb{k}'}}} \delta_{\vb{k},-\vb{k}'} (\omega_{\vb{k}'} - \omega_{\vb{k}})
\end{align*}

However, $\omega_{\vb{k}} = \omega_{-\vb{k}}$. Hence, when the delta function is non-zero (i.e. $\vb{k} = -\vb{k}'$), $\omega_{\vb{k}}-\omega_{\vb{k}'} = 0$. Therefore,
\begin{gather*}
	\left[a(\vb{k}), a(\vb{k}')\right] = 0 \\
	\intertext{and, similarly,}
	\left[a^\dagger(\vb{k}), a^\dagger(\vb{k}')\right] = 0		 
\end{gather*}
\end{problem}



\begin{problem}
A pair of complex Klein-Gordon fields $\phi(x)$ and $\phi^\dagger(x)$ can be described by two independent real fields $\phi_r(x), r=1,2$:
\begin{equation*}
	\phi = \frac{1}{\sqrt{2}}\left(\phi_1 + i\phi_2\right) \,, \,
	\phi^\dagger = \frac{1}{\sqrt{2}}\left(\phi_1 - i\phi_2\right)
\end{equation*}
and each of the real fields can be expanded as:
\begin{equation*}
	\phi_r(x) = \sum_{\vb{k}} \sqrt{\frac{\hbar c^2}{2V \omega_{\vb{k}}}} \left[a_r(\vb{k})e^{-ikx} + a^\dagger_r(\vb{k})e^{ikx}\right]
\end{equation*}

Now, suppose we wish to expand the complex field $\phi(x)$ as:
\begin{equation*}
	\phi(x) = \sum_{\vb{k}} \sqrt{\frac{\hbar c^2}{2V \omega_{\vb{k}}}} \left[a(\vb{k})e^{-ikx} + b^\dagger(\vb{k})e^{ikx}\right]
\end{equation*}
This expansion, in terms of the real fields $\phi_r(x)$, requires:
\begin{align*}
	&= \sum_{\vb{k}} \sqrt{\frac{\hbar c^2}{2V \omega_{\vb{k}}}} \frac{1}{\sqrt{2}}\left[a_1(\vb{k})e^{-ikx} + a^\dagger_1(\vb{k})e^{ikx} + ia_2(\vb{k})e^{-ikx} + ia^\dagger_2(\vb{k})e^{ikx}\right]	\\
	&= \sum_{\vb{k}} \sqrt{\frac{\hbar c^2}{2V \omega_{\vb{k}}}}
	\left[\frac{1}{\sqrt{2}}\big(a_1(\vb{k}) + ia_2(\vb{k})\big)e^{-ikx} + \frac{1}{\sqrt{2}}\big(a^\dagger_1(\vb{k}) + ia^\dagger_2(\vb{k})\big)e^{ikx}\right]
\end{align*}

Therefore, this expansion of the complex field $\phi(x)$ requires:
\begin{align*}
	a(\vb{k}) = \frac{1}{\sqrt{2}}\left(a_1(\vb{k}) + ia_2(\vb{k})\right)
\end{align*}
and
\begin{align*}
	b(\vb{k}) = \frac{1}{\sqrt{2}}\left(a_1(\vb{k}) - ia_2(\vb{k})\right)		 
\end{align*}

From here on, let $x=(\vb{x},t)$ and $x'=(\vb{x}',t)$. In other words, we will not consider commutators of operators at different times in this problem.

Now, from the commutator relationships of the real field operators, we can compute those of the complex field operators.
\begin{align*}
	\left[\phi(x), \dot{\phi}^\dagger(x')\right]
	&= \left[\frac{1}{\sqrt{2}}\left(\phi_1(x)+i\phi_2(x)\right), \frac{1}{\sqrt{2}}\left(\dot{\phi}_1(x') - i\dot{\phi}_2(x')\right)\right]	\\
	&= \frac{1}{2}\left(\left[\phi_1(x), \dot{\phi}_1(x')\right] + \left[\phi_2(x), \dot{\phi}_2(x')\right]\right)	\\
	&= i\hbar c^2 \delta\left(\vb{x}-\vb{x}'\right)
\end{align*}
since $\left[\phi_r(x), \phi_r(x')\right] = i \hbar c^2 \delta\left(\vb{x}-\vb{x}'\right)$ for any real field $\phi_r$.

As one would expect based on the commutators of real fields, we see that:
\begin{align*}
	\big[\phi(x), \phi(x')\big] &= \big[\dot{\phi}(x), \dot{\phi}(x')\big]	\\
	= \big[\phi^\dagger(x), \phi^\dagger(x')\big] &= \big[\dot{\phi}^\dagger(x), \dot{\phi}^\dagger(x')\big] = 0
\end{align*}
Also, because $\phi_1, \phi_2$ commute (at a given time) and $\phi, \phi^\dagger$ are merely the linear combinations, we also see that:
\begin{equation*}
	\left[\phi(x), \phi^\dagger(x')\right] = \left[\dot{\phi}(x), \dot{\phi}^\dagger(x')\right] = 0
\end{equation*}
(In this sense, $\phi$ and $\phi^\dagger$ are independent fields.)

One less intuitive (and different from the properties of real fields) result is that:
\begin{align*}
	\left[\phi(x), \dot{\phi}(x')\right]
	&= \left[\frac{1}{\sqrt{2}}\left(\phi_1(x)+i\phi_2(x)\right), \frac{1}{\sqrt{2}}\left(\dot{\phi}_1(x') + i\dot{\phi}_2(x')\right)\right]	\\
	&= \frac{1}{2} \left(\left[\phi_1(x), \dot{\phi}_1(x')\right] - \left[\phi_2(x), \dot{\phi}_2(x')\right]\right)	\\
	&= 0
\end{align*}
and, similarly, $\left[\phi^\dagger(x), \dot{\phi}^\dagger(x')\right]=0$.

Furthermore, we see that:
\begin{align*}
	\left[a(\vb{k}), a^\dagger(\vb{k}')\right]
	&= \left[\frac{1}{\sqrt{2}}\big(a_1(\vb{k}) + ia_2(\vb{k})\big), \frac{1}{\sqrt{2}}\big(a^\dagger_1(\vb{k}')-ia^\dagger_2(\vb{k}')\big)\right]	\\
	&= \frac{1}{2}\Big(\big[a_1(\vb{k}), a^\dagger_1(\vb{k}')\big] + \big[a_2(\vb{k}), a^\dagger_2(\vb{k}')\big]\Big)	\\
	&= \delta_{\vb{k}\vb{k}'}
\end{align*}
and, similarly, $\left[b(\vb{k}), b^\dagger(\vb{k}')\right] = \delta_{\vb{k}, \vb{k}'}$.

It is easy to see that
\begin{equation*}
	\big[a(\vb{k}), a(\vb{k}')\big] = \big[b(\vb{k}), b(\vb{k}')\big] = \big[a(\vb{k}), b(\vb{k}')\big] = 0
\end{equation*}
since these will involve annihilation operators \emph{only}, and the two real fields are independent. The same holds for their Hermitian conjugates (creation operators only).

Lastly, we also can compute:
\begin{align*}
	\big[a(\vb{k}), b^\dagger(\vb{k}')\big]
	&= \left[\frac{1}{\sqrt{2}}\big(a_1(\vb{k}) + ia_2(\vb{k})\big), \frac{1}{\sqrt{2}}\big(a^\dagger_1(\vb{k}') + ia^\dagger_2(\vb{k}')\big)\right]	\\
	&= \frac{1}{2}\Big(\big[a_1(\vb{k}), a^\dagger_1(\vb{k}')\big] - \big[a_2(\vb{k}), a^\dagger_2(\vb{k}')\big]\Big)	\\
	&= 0		 
\end{align*}
\end{problem}



\begin{problem}
The Feynman $\Delta$-function can be written as:
\begin{equation*}
	\Delta_\text{F}(x) = \frac{1}{(2\pi)^4} \int\diff^4{k} \, \frac{e^{-ikx}}{k^2-\mu^2+i\epsilon}
\end{equation*}
where we let $\epsilon$ tend to zero after the integration.

Applying the operator $\left(\dalem+\mu^2\right)$ to this function, we get:
\begin{align*}
	\left(\dalem + \mu^2\right) \Delta_\text{F}(x)
	&= \left(\dalem + \mu^2\right) \frac{1}{(2\pi)^4} \int\diff[4]{k} \, \frac{e^{-ikx}}{k^2-\mu^2+i\epsilon}	\\
	&= \frac{1}{(2\pi)^4} \int\diff[4]{k}\, \frac{1}{k^2-\mu^2+i\epsilon}\left(\dalem + \mu^2\right)e^{-ikx}	\\
	&= \frac{1}{(2\pi)^4} \int\diff[4]{k} \, \frac{1}{k^2-\mu^2+i\epsilon} \left(-k^2+\mu^2\right)e^{-ikx}	\\
	&= \frac{1}{(2\pi)^4} \int\diff[4]{k} \, \frac{-e^{-ikx}}{1+i\kappa}	\\
	&= -\delta^{(4)}(x)			 
\end{align*}
where we have defined $\kappa = \epsilon/(k^2-\mu^2)$ and let it tend to zero.

Hence, the Feynman $\Delta$-function satisfies the inhomogeneous Klein-Gordon equation.
\end{problem}



\begin{problem}
Let $\phi$ be a complex Klein-Gordon field, representing a charged meson. It can be written as:
\begin{equation*}
	\phi(x) = \phi^+(x) + \phi^-(x) = \sum_{\vb{k}} \sqrt{\frac{\hbar c^2}{2V\omega_{\vb{k}}}} \big[a(\vb{k})e^{-ikx} + b^\dagger(\vb{k})e^{ikx}\big]
\end{equation*}
where the operators $a$ and $b$ follow the commutator relationships derived earlier.

Recall from Problem~2 that:
\begin{equation*}
	\big[a(\vb{k}), a(\vb{k}')\big] = \big[b^\dagger(\vb{k}), b^\dagger(\vb{k}')\big] = \big[a(\vb{k}), b^\dagger(\vb{k}')\big] = 0
\end{equation*}
and therefore, unlike the commutator of a real field, we now have:
\begin{equation*}
	\left[\phi(x), \phi(x')\right] = 0
\end{equation*}
for any two different points $x,x'$ in spacetime.

Instead, for complex fields, we see that:
\begin{align*}
	\big[\phi(x), \phi^\dagger(x')\big]
	&= \big[\phi^+(x), \phi^{\dagger-}(x')\big] + \big[\phi^-(x), \phi^{\dagger+}(x')\big]
\end{align*}
is non-zero.

The first commutator is:
\begin{align*}
	\big[\phi^+(x), \phi^{\dagger-}(x')\big]
	&= \frac{\hbar c^2}{2V} \sum_{\vb{k}, \vb{k}'} \frac{1}{\sqrt{\omega_{\vb{k}} \omega_{\vb{k}'}}} \big[a(\vb{k}), a^\dagger(\vb{k}')\big]e^{-i\left(kx-k'x\right)}	\\
	&= \frac{\hbar c^2}{2V} \sum_{\vb{k}, \vb{k}'} \frac{1}{\sqrt{\omega_{\vb{k}} \omega_{\vb{k}'}}} \delta_{\vb{k}\vb{k}'} e^{-i\left(kx-k'x\right)}	\\
	&= \frac{\hbar c^2}{2V} \sum_{\vb{k}}	\frac{e^{-ik(x-x')}}{\omega_{\vb{k}}}
\intertext{which, upon taking the limit $V\rightarrow\infty$, becomes:}
	&= \frac{\hbar c^2}{2(2\pi)^3} \int \diff[3]{\vb{k}} \, \frac{e^{-ik(x-x')}}{\omega_{\vb{k}}}	\\
	&\define i\hbar c \Delta^+(x-x')
\end{align*}
where
\begin{equation*}
	\Delta^+(x-x') \define \frac{-ic}{2(2\pi)^3}\int\frac{\diff[3]{\vb{k}}}{\omega_{\vb{k}}} e^{-ik(x-x')}
\end{equation*}
is defined the same way as it was defined for real field Klein-Gordon fields (neutral mesons). Similarly, the second commutator is:
\begin{align*}
	\big[\phi^-(x), \phi^{\dagger+}(x')\big]
	&= i\hbar c \Delta^-(x-x')
\end{align*}

Therefore, we have:
\begin{equation*}
	\big[\phi(x), \phi^\dagger(x')\big] = i\hbar c \left(\Delta^+(x-x') + \Delta^-(x-x')\right) \define i\hbar c \Delta(x-x')
\end{equation*}
where $\Delta(x) \define \Delta^+(x) + \Delta^-(x)$.

Now, define the Feynman $\Delta$-function of a complex Klein-Gordon field $\phi$ as:
\begin{equation*}
	i\hbar c \Delta_\text{F}(x-x') \define \ev{\timep{\phi(x)\phi^\dagger(x')}}{0}
\end{equation*}
where $\timep{\hspace{1em}}$ is the time-ordered product.

Then, this Feynman $\Delta$-function, or \emph{the Feynman propagator}, turns out to be of the same form as that of a real Klein-Gordon field:
\begin{align*}
	\Delta_\text{F}(x) &= \theta(t)\Delta^+(x) - \theta(-t)\Delta^-(x)		\\
	&= \frac{1}{(2\pi)^4} \int \diff[4]{k} \frac{e^{-ikx}}{k^2-\mu^2+i\epsilon}		 
\end{align*}
where $\theta(t)$ is the Heaviside step function.

If $t'<t$, then $\timep{\phi(x)\phi^\dagger(x')} = \phi(x)\phi^\dagger(x')$, and the Feynman propagator becomes:
\begin{align*}
	i\hbar c \Delta_\text{F}(x-x') &= \ev{\phi(x)\phi^\dagger(x')}{0}	\\
	&= \ev{\phi^+(x)\phi^{\dagger-}(x')}{0}
\end{align*}
which is interpreted as: a particle is created by $a^\dagger(\vb{k})$ at $x'$, then propagates to $x$ which is a later point in time, and is absorbed (annihilated) by $a(\vb{k})$ at $x$.

When $t'>t$, then $\timep{\phi(x)\phi^\dagger(x')} = \phi^\dagger(x')\phi(x)$, and therefore:
\begin{align*}
	i\hbar c \Delta_\text{F}(x-x') &= \ev{\phi^\dagger(x')\phi(x)}{0}	\\
	&= \ev{\phi^{\dagger+}(x')\phi^-(x)}{0}
\end{align*}
which is interpreted as: an anti-particle is created by $b^\dagger(\vb{k})$ at $x$, then propagates to $x'$, and is absorbed (annihilated) by $b(\vb{k})$ at $x$.
\end{problem}



\begin{problem}
The charge conjugation operator $\mathscr{C}$ is defined by:
\begin{equation*}
	\phi(x) \longrightarrow \mathscr{C}\phi(x)\mathscr{C}^{-1} = \eta_\text{c} \, \phi^\dagger(x)
\end{equation*}
where $\mathscr{C}$ is a unitary operator which leaves the vacuum invariant (i.e., $\mathscr{C}\ket{0} = \ket{0}$), and $\eta_\text{c}$ is a phase factor.

First, note that:
\begin{align*}
	\phi^\dagger(x) &\rightarrow \mathscr{C}\phi^\dagger(x)\mathscr{C}^{-1}	\\
\intertext{and, since $\mathscr{C}$ is unitary, we can rewrite this as:}
	&= (\mathscr{C}^{-1})^\dagger \phi^\dagger(x) C^\dagger	\\
	&= \left(\mathscr{C} \phi(x) \mathscr{C}^{-1}\right)^\dagger	\\
	&= \left(\eta_\text{c} \phi^\dagger(x)\right)^\dagger	\\
	&= \eta_\text{c}^* \, \phi(x)
\end{align*}

Then, under the charge conjugation, the complex Klein-Gordon Lagrangian density $\lagr = \mathrm{N}\Big((\partial_\alpha \phi^\dagger) (\partial^\alpha \phi) - \mu^2 \phi^\dagger\phi\Big)$ becomes:
\begin{align*}
	\lagr' &= \mathrm{N}\Big((\eta^*_\text{c}\partial_\alpha \phi) (\eta_\text{c}\partial^\alpha \phi^\dagger) - \mu^2 (\eta^*_\text{c}\phi)(\eta_\text{c}\phi^\dagger) \Big)	\\
	&= \left(\eta^*_\text{c} \eta_\text{c}\right)
	\mathrm{N}\Big((\partial_\alpha \phi)(\partial^\alpha \phi^\dagger) - \mu^2 \phi\phi^\dagger \Big)	\\
	\intertext{Since all commutators inside the normal product vanish, this simplifies to:}
	&= \mathrm{N}\Big((\partial_\alpha \phi^\dagger) (\partial^\alpha \phi) - \mu^2 \phi^\dagger\phi\Big)		\\
	&= \lagr			 
\end{align*}
Thus, the Lagrangian density is invariant under the charge conjugation.


On the other hand, the charge-current density
\begin{equation*}
	s^\alpha = -i\frac{q}{\hbar} \mathrm{N}\Big(\phi \partial^\alpha\phi^\dagger - \phi^\dagger \partial^\alpha\phi\Big)
\end{equation*}
becomes:
\begin{align*}
	s'^\alpha &= -i\frac{q}{\hbar} \mathrm{N}\Big((\eta_\text{c}\phi^\dagger)\partial^\alpha(\eta^*_\text{c}\phi) - (\eta^*_\text{c}\phi)\partial^\alpha(\eta_\text{c}\phi^\dagger)\Big)	\\
	&= -i\frac{q}{\hbar} (\eta^*_\text{c}\eta_\text{c}) \mathrm{N}\Big(\phi^\dagger\partial^\alpha \phi - \phi \partial^\alpha \phi^\dagger\Big)	\\
	&= +i \frac{q}{\hbar} \mathrm{N}\Big(\phi \partial^\alpha\phi^\dagger - \phi^\dagger \partial^\alpha\phi\Big)		\\
	&= -s^\alpha			 
\end{align*}
Thus, the charge-current density changes sign under the charge conjugation.

Writing $\phi(x), \phi^\dagger(x)$ out in their expansion forms in terms of the absorption and creation operators, we see that:
\begin{align*}
	\mathscr{C} \phi(x) \mathscr{C}^{-1}
	&= \sum_{\vb{k}} \sqrt{\frac{\hbar c^2}{2V\omega_{\vb{k}}}} \big[\mathscr{C} a(\vb{k}) \mathscr{C}^{-1} e^{-ikx} + \mathscr{C} b^\dagger(\vb{k}) \mathscr{C}^{-1} e^{ikx}\big]	\\
	&= \eta_\text{c} \, \phi^\dagger(x)	\\
	&= \eta_\text{c} \sum_{\vb{k}} \sqrt{\frac{\hbar c^2}{2V \omega_{\vb{k}}}} \big[a^\dagger(\vb{k})e^{ikx} + b(\vb{k})e^{-ikx}\big]
\end{align*}
which implies that:
\begin{equation*}
	\mathscr{C} a(\vb{k}) \mathscr{C}^{-1} = \eta_\text{c}\, b(\vb{k})	\,,
	\, \mathscr{C} b(\vb{k}) \mathscr{C}^{-1} = \eta^*_\text{c} \, a(\vb{k})
\end{equation*}

Then, we also see that, under the charge conjugation, single-particle states transform as:
\begin{align*}
	\mathscr{C} \ket{a,\vb{k}} &= \mathscr{C} a^\dagger(\vb{k}) \ket{0}	\\
	&= \mathscr{C} a^\dagger(\vb{k}) \mathscr{C}^{-1} \ket{0}	\\
	&= \left(\mathscr{C}a(\vb{k})\mathscr{C}^{-1}\right)^\dagger \ket{0}	\\
	&= \eta^*_\text{c} \, b^\dagger(\vb{k})\ket{0}	\\
	&= \eta^*_\text{c} \ket{b,\vb{k}}
\intertext{and, similarly,}
	\mathscr{C} \ket{b,\vb{k}} &= \eta_\text{c} \ket{a,\vb{k}}		 
\end{align*}
where we have used the fact that the vacuum state $\ket{0}$ is unchanged by $\mathscr{C}$ or $\mathscr{C}^{-1}$.

Thus, the charge conjugation $\mathscr{C}$ \emph{interchanges} the particles ($a$-particles) and the anti-particles ($b$-particles).
\end{problem}



\begin{problem}
The parity transformation (i.e. spatial inversion) of the Hermitian (real) Klein-Gordon field $\phi(x)$ is defined by:
\begin{equation*}
	\phi(\vb{x},t) \longrightarrow \mathscr{P}\phi(\vb{x},t)\mathscr{P}^{-1} = \eta_\text{p} \, \phi(-\vb{x},t)
\end{equation*}
where the parity operator $\mathscr{P}$ is unitary which leaves the vacuum invariant (i.e., $\mathscr{P}\ket{0} = \ket{0}$), and $\eta_\text{p} = \pm1$ is called \emph{the intrinsic parity} of the field. Note that $\mathscr{P}$ \emph{only} influences the spatial argument of the field, but not the time $t$.

Under the given parity transformation, the Lagrangian density of the real Klein-Gordon field $\lagr = \frac{1}{2} \left((\partial_\alpha \phi)(\partial^\alpha \phi) - \mu^2\phi^2\right)$ transform as:
\begin{align*}
	\lagr'(\vb{x},t) &= \frac{1}{2} \left[\left(\eta_\text{p}\pder{\phi(-\vb{x},t)}{x^\alpha}\right)\left(\eta_\text{p}\pder{\phi(-\vb{x},t)}{x_\alpha}\right) - \mu^2 \eta_\text{p}^2 \phi^2(-\vb{x},t) \right]	\\
	&= \frac{\eta_\text{p}^2}{2}\left[\frac{1}{c^2}\npder{2}{\phi(-\vb{x},t)}{t} - \left(\grad{\phi(-\vb{x},t)}\right)^2 - \mu^2\phi^2(-\vb{x},t)\right]	\\
	&= \frac{1}{2} \left[\frac{1}{c^2}\npder{2}{\phi(\vb{x},t)}{t} - (-1)^2(\grad{\phi(\vb{x},t)})^2 - \mu^2\phi^2(-\vb{x},t)\right]	\\
	&= \frac{1}{2} \left[(\partial_\alpha \phi(\vb{x},t))(\partial^\alpha \phi(\vb{x,}t)) - \mu^2\phi^2(-\vb{x},t)\right]
\end{align*}
where we have used the chain rule for $\grad$ and the fact that $\eta_\text{p}^2=+1$.

Also, $\phi^2$ is necessarily an even function of each component of $x$. i.e.,
\begin{equation*}
	\phi^2(\vb{x},t) = \phi^2(-\vb{x},t)
\end{equation*}

Thus, the Lagrangian density $\lagr$ is invariant under the parity transformation:
\begin{equation*}
	\lagr'(\vb{x},t) = \frac{1}{2}\left[(\partial_\alpha \phi(\vb{x},t))(\partial^\alpha \phi(\vb{x},t)) - \mu^2\phi^2(\vb{x},t)\right] = \lagr(\vb{x},t)		 
\end{equation*}

Next, using the field operator $\phi(x)$ in the expansion form in terms of the absorption and creation operators, we can write:
\begin{align*}
	\phi(-\vb{x},t)	&= \sum_{\vb{k}'} \sqrt{\frac{\hbar c^2}{2V \omega_{\vb{k}'}}} \big[a(\vb{k}')e^{-i(\omega_{\vb{k}'}t + \vb{k}'\dotp\vb{x})} + a^\dagger(\vb{k}')e^{i(\omega_{\vb{k}'}t + \vb{k}'\dotp\vb{x})}\big]
\intertext{Define $\vb{k} = -\vb{k}'$. Note that $\omega_{\vb{k}'} = \omega_{-\vb{k}} =\omega_{\vb{k}}$. Then, we can write:}
	&= \sum_{\vb{k}} \sqrt{\frac{\hbar c^2}{2V\omega_{\vb{k}}}}
	\big[a(-\vb{k})e^{-i(\omega_{\vb{k}}t - \vb{k}\dotp\vb{x})} + a^\dagger(-\vb{k})e^{i(\omega_{\vb{k}}t - \vb{k}\dotp\vb{x})}\big]
\end{align*}

Then, the parity transformation of $\phi(x)$ can be written as:
\begin{align*}
	\mathscr{P} \phi(\vb{x},t) \mathscr{P}^{-1}
	&= \sum_{\vb{k}} \sqrt{\frac{\hbar c^2}{2V \omega_{\vb{k}}}}
	\big[\mathscr{P}a(\vb{k})\mathscr{P}^{-1}e^{-i(\omega_{\vb{k}}t - \vb{k}\dotp\vb{x})}
	+ \mathscr{P}a^\dagger(\vb{k})\mathscr{P}^{-1}e^{i(\omega_{\vb{k}}t - \vb{k\dotp\vb{x}})}\big]	\\
	&= \eta_\text{p} \sum_{\vb{k}} \sqrt{\frac{\hbar c^2}{2V\omega_{\vb{k}}}}
	\big[a(-\vb{k})e^{-i(\omega_{\vb{k}}t - \vb{k}\dotp\vb{x})} + a^\dagger(-\vb{k})e^{i(\omega_{\vb{k}}t - \vb{k}\dotp\vb{x})}\big]
\end{align*}

Therefore, we derive the transformation laws for the absorption and creation operators under $\mathscr{P}$:
\begin{equation*}
	\mathscr{P} a(\vb{k}) \mathscr{P}^{-1} = \eta_\text{p} \, a(-\vb{k})	\,, \, \mathscr{P} a^\dagger(\vb{k}) \mathscr{P}^{-1} = \eta_\text{p} \, a^\dagger(-\vb{k})
\end{equation*}

Then, we can derive how an arbitrary $n$-particle state transforms:
\begin{align*}
	\mathscr{P} \ket{\vb{k}_1, ... \,, \vb{k}_n}
	&= \mathscr{P} a^\dagger(\vb{k}_1) \,... \, a^\dagger(\vb{k}_n) \ket{0}	\\
	&= \mathscr{P} a^\dagger(\vb{k}_1) \,... \, a^\dagger(\vb{k}_n) \mathscr{P}^{-1} \ket{0}	\\
	&= \left(\mathscr{P}a^\dagger(\vb{k}_1)\mathscr{P}^{-1}\right)\,...\,
	\left(\mathscr{P}a^\dagger(\vb{k}_n)\mathscr{P}^{-1}\right) \ket{0}	\\
	&= \eta_\text{p}^n a^\dagger(-\vb{k}_1) \, ... \, a^\dagger(-\vb{k}_n) \ket{0}	\\
	&= \eta_\text{p}^n \ket{-\vb{k}_1, ... \,, -\vb{k}_n}		 
\end{align*}

Now, let $A$ and $B$ arbitrary operators, and define:
\begin{equation*}
	B_0=B , \quad B_n =\left[A, B_{n-1}\right] , \quad \text{for } n=1, 2, ...
\end{equation*}
Then, the following holds identically:
\begin{equation*}
	e^{i\alpha A} B e^{-i\alpha A} = \sum_{n=0}^{\infty} \frac{(i\alpha)^n}{n!} B_n
\end{equation*}

We choose $B = B_0 = a(\vb{k})$.

\begin{subproblem}
	First, we define:
	\begin{gather*}
		\mathscr{P}_1 \define e^{i\alpha_1 A_1}	\\
	\intertext{where}
		\alpha_1 = -\frac{\pi}{2} , \quad A_1 = \sum_{\vb{k}}a^\dagger(\vb{k})a(\vb{k})
	\end{gather*}
	Then, we can calculate the commutators $[A_1, B_{n-1}]$:
	\begin{align*}
		[A_1, B_0] &= \left[\sum_{\vb{k}'} a^\dagger(\vb{k}')a(\vb{k}'), a(\vb{k})\right]	\\
		&= \sum_{\vb{k}'} \left[a^\dagger(\vb{k}'), a(\vb{k})\right]a(\vb{k}')	\\
		&= -\sum_{\vb{k}'} \delta_{\vb{k}'\vb{k}} a(\vb{k}')	\\
		&= -a(\vb{k})
	\end{align*}
	and, by induction,
	\begin{equation*}
		\left[A_1, B_{n-1}\right] = (-1)^n \, a(\vb{k})
	\end{equation*}
	
	Therefore, we get:
	\begin{align*}
		\mathscr{P}_1 a(\vb{k}) \mathscr{P}_1^{-1}
		&= e^{i\alpha_1A_1} a(\vb{k}) e^{-i\alpha_1A_1}	\\
		&= \sum_{n=0}^{\infty} \frac{1}{n!} \left(+i\frac{\pi}{2}\right)^n \, a(\vb{k})	\\
		&= e^{i\pi/2} a(\vb{k})	\\
		&= i a(\vb{k})
	\end{align*}
\end{subproblem}

\begin{subproblem}
	Next, we define:
	\begin{gather*}
		\mathscr{P}_2 \define e^{i\alpha_2 A_2} 	\\
	\intertext{where}
		\alpha_2 = \frac{\pi}{2}\eta_\text{p} , \quad
		A_2 = \sum_{\vb{k}} a^\dagger(\vb{k}) a(-\vb{k})
	\end{gather*}
	Then, the commutators $[A_2, B_{n-1}]$ are:
	\begin{align*}
		[A_2, B_{n_1}] &= (-1)^n \, a(-\vb{k})
	\end{align*}
	
	Therefore, we get:
	\begin{align*}
		\mathscr{P}_2 a(\vb{k}) \mathscr{P}_2^{-1}
		&= e^{i\alpha_2A_2} a(\vb{k}) e^{-i\alpha_2A_2}	\\
		&= \sum_{n=0}^{\infty} \frac{1}{n!} \left(-i\frac{\pi}{2}\eta_\text{p}\right)^n a(-\vb{k})	\\
		&= e^{-i\eta_\text{p}\pi/2} a(-\vb{k})	\\
		&= -i \eta_\text{p} a(-\vb{k})
	\end{align*}
\end{subproblem}

Since $\alpha_r$'s and $A_r$'s are real for both $r=1,2$, $\mathscr{P}_r = e^{i\alpha_r A_r}$ are unitary. Therefore, the product $\mathscr{P}_1 \mathscr{P}_2$ is unitary as well. Note that the product operator satisfies the following identity:
\begin{align*}
	\left(\mathscr{P}_1\mathscr{P}_2\right) a(\vb{k}) \left(\mathscr{P}_1\mathscr{P}_2\right)^{-1}
	&= \mathscr{P}_1 \mathscr{P}_2 a(\vb{k}) \mathscr{P}_2^{-1} \mathscr{P}_1^{-1}	\\
	&= \mathscr{P}_1 \left(-i\eta_\text{p}a(-\vb{k})\right) \mathscr{P}_1^{-1}	\\
	&= +\eta_\text{p} a(-\vb{k})		 
\end{align*}
which was derived earlier in this problem as the parity transformation law for the absorption operator $a(\vb{k})$.

Thus, $\mathscr{P}_1\mathscr{P}_2$ gives an explicit form for the parity operator $\mathscr{P}$.
\end{problem}

\pagebreak



\section{The Dirac Field}

\begin{problem}
For any given Dirac field $\psi$, we have the anti-commutator relationship:
\begin{equation*}
	\big\{\psi(x), \overbar{\psi}(y)\big\} \define \big[\psi(x), \overbar{\psi}(y)\big]_+ = i S(x-y)
\end{equation*}
where
\begin{gather*}
	S(x) = S^+(x) + S^-(x) = \left(i\gamma^\mu\pder{}{x^\mu} + \frac{mc}{\hbar}\right) \Delta(x)	\\
	\Delta(x) = \Delta^+(x) + \Delta^-(x)
	= \frac{-c}{(2\pi)^3} \int \frac{\diff[3]{\vb{k}}}{\omega_{\vb{k}}} \sin(kx)
\end{gather*}

First, note that we have:
\begin{align*}
	\pder{}{x^\mu} \Delta(x) &= \frac{-c}{(2\pi)^3} \int \frac{\diff[3]{\vb{k}}}{\omega_{\vb{k}}} \pder{}{x^\mu} \sin(k_\nu x^\nu)	\\
	&= \frac{-c}{(2\pi)^3} \int \frac{\diff[3]{\vb{k}}}{\omega_{\vb{k}}} k_\mu \cos\left(\omega_{\vb{k}}t - \vb{k}\dotp\vb{x}\right)
\end{align*}

Now, consider two points $x,y$ at a same given time. That is,
\begin{equation*}
	x=\left(ct, \vb{x}\right) \text{ and } y=\left(ct,\vb{y}\right)
\end{equation*}
Then, the anti-commutator above becomes:
\begin{align*}
	\left.\big\{\psi(x), \overbar{\psi}(y) \big\}\right|_{x^0=y^0}
	&= iS(0, \vb{x}-\vb{y})	\\
	&= \frac{-ic}{(2\pi)^3} \int \frac{\diff[3]{\vb{k}}}{\omega_{\vb{k}}}
	\Big[i\gamma^\mu k_\mu \cos\left(-\vb{k}\dotp(\vb{x}-\vb{y})\right) + \sin\left(-\vb{k}\dotp(\vb{x}-\vb{y})\right)\Big]
\end{align*}
Note that, if $\vb{x}-\vb{y} \neq \vb{0}$, the integrand is an odd function of $\vb{k}$. Therefore, the anti-commutator vanishes if $x^0=y^0$ and $x^i \neq y^i$. However, if $x=y$, the above expression becomes:
\begin{align*}
	\frac{-ic}{(2\pi)^3} \int \frac{\diff[3]{\vb{k}}}{\omega_{\vb{k}}}
	\Big[i\gamma^\mu k_\mu \cos0 + \sin0\Big]
	&= 	\frac{-ic}{(2\pi)^3} \int \frac{\diff[3]{\vb{k}}}{\omega_{\vb{k}}}
	i\Big[\gamma^0\frac{\omega_{\vb{k}}}{c} - \gamma^i k_i\Big]	\\
	&= +\gamma^0 \frac{1}{(2\pi)^3} \int \diff[3]{\vb{k}}
\end{align*}
where the odd part of the integrand $\gamma^i k_i$ vanishes again.

Therefore, the equal-time commutator for the Dirac field $\psi$ is:
\begin{equation*}
	\big\{\psi(x), \overbar{\psi}(y)\big\} \define \big[\psi(x), \overbar{\psi}(y)\big]_+ = \gamma^0 \delta(\vb{x}-\vb{y})
\end{equation*}
\end{problem}



\begin{problem}
\begin{align*}
	\left(i\gamma^\mu\pder{}{x^\mu} - \frac{mc}{\hbar}\right) S(x)
	&= \left(i\gamma^\mu\pder{}{x^\mu} - \frac{mc}{\hbar}\right) \left(i\gamma^\nu\pder{}{x^\nu} + \frac{mc}{\hbar}\right) \Delta(x)	\\
	&= \left(-\gamma^\mu \gamma^\nu \partial_\mu \partial_\nu  + \left(\frac{mc}{\hbar}\right)^2\right) \Delta(x)	\\
\intertext{Recall that $\gamma^\mu\gamma^\nu = g^{\mu\nu}$. Defining $\mu\define mc/\hbar$, we get:}
	&= -\left(\dalem + \mu^2 \right)\Delta(x)	\\
	&= 0 
\end{align*}
because $\Delta(x)$ satisfies the Klein-Gordon equation, which was mentioned in Chapter~3.

Similarly, we also have:
\begin{align*}
	\left(i\gamma^\mu\pder{}{x^\mu} - \frac{mc}{\hbar}\right) S_\text{F}(x)
	&= \left(i\gamma^\mu\pder{}{x^\mu} - \frac{mc}{\hbar}\right) \left(i\gamma^\nu\pder{}{x^\nu} + \frac{mc}{\hbar}\right) \Delta_\text{F}(x)	\\
	&= -\left(\dalem + \mu^2 \right)\Delta_\text{F}(x)	\\
	&= +\delta^{(4)}(x)
\end{align*}
where the last step is the result of Chapter~3 Problem~3.

Hence, the functions $S(x)$ and $S_\text{F}(x)$ satisfy the homogeneous and inhomogeneous Dirac equations respectively (because $\Delta(x)$ and $\Delta_\text{F}(x)$ satisfy the homogeneous and inhomogeneous Klein-Gordon equations).
\end{problem}




\end{document}
